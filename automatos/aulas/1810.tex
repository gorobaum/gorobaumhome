\documentclass{article}
\usepackage{amsmath}
\begin{document}

\section{4 - Lema do Bombeamento (ou Lema da iteracao) para linguagens reconheciveis}

Quais das linguagens a seguir sao reconheciveis?

\begin{align*}
A = \{a^ib^i : i \geq 0\} \\
B = \{\omega \in  \{a,b\}^* \colon |\omega|_a = |\omega|_b\} \\
C = \{\omega \in  \{a,b\}^* \colon O n de ocorrencias do fator \emph{ab} em \omega e 
        igual ao n de ocorrencias do fator \emph{ba} em \omega \}
\end{align*}

\begin{itemize}
\item A linguagem A nao e reconhecivel. \\
    Suponha que A seja reconhecivel. \\
    Entao, existe um \emph{afd} $\mathcal{A} = (Q, \Sigma, \delta, s, F)$ tq $L(\mathcal{A}) = A.$ \\
    Seja $ n = |Q| $ a ocorrencia 

    ----TODO OMI----


\subsection{Lema do Bombeamento}
    Seja L uma linguagem reconhecivel.\\
    Entao, existe um inteiro $n \ge 1$ tq para cada palavra $\omega \in L$, com 
    $|\omega| \ge n$, existem palavras $x$, $y$ e $z$ tq $x = xyz$, $y \neq \lambda$,
    $|xy| \le n$ e para todo $k \ge 0$, $xy^kz \in L$.

    Prova: \\
    Seja L uma linguagem reconhecivel.\\
    Entao, existe um \emph{afd} $ \mathcal{A} = (Q, \Sigma, \delta, s, F)$ tq $L(\mathcal{A}) = L.$\\
    Considere $n = |Q|$.\\
    Seja em L nao existem palavras de comprimento $\ge n$, nada ha para provar.\\
    Caso contrario, seja $\omega \in L$, com $|\omega| \ge n$.\\
    Entao, $w = \sigma_1\sigma_2\dots\sigma_n\omega`$, com $\sigma_i \in \Sigma$ ( para $1 \le i \le n$ )
    e $\omega \in \Sigma^+$
\end{itemize}
\end{document}
