\documentclass{article}
\usepackage[brazilian]{babel}
\usepackage[utf8]{inputenc}
\usepackage{amsmath}
\usepackage{amssymb}
\usepackage{hyperref}
\begin{document}

\newtheorem{thm}{Theorem}

\begin{itemize}
    \item Uma linguagem l.c. $L$ é não-ambígua se existe uma glc não-ambígua $G$ tq $ L = L(G) $.
    \item Uma linguagem l.c. $L$ é inerentemente ambígua se toda glc que gera $L$ é ambígua.\\
            \underline{Exemplo}: $ L = \{ a^ib^jc^k : i,j,k \geq 0 $ e $ ( i = j $ ou $ j = k ) \} $
\end{itemize}
    
    \underline{Obs}: As palavras $ a^nb^nc^n $, para $ n \geq 0 $, são deriváveis de duas maneiras diferentes.
    
    \underline{Exercício}: Escreva uma glc que gera $L$.
    
    \underline{Teorema}: Não existe um algoritmo para decidir se uma linguagem livre de contexto é
                            inerentemente ambígua ou não.\\
                            
    \underline{Teorema 4}: A classe das ling. l.c. é fechada para as operações de união, concatenação
                            e estrela.
                            
    \underline{Prova}: Sejam $L_1$ e $L_2$ ling. l.c. sobre $\Sigma$. \\
        Então, existem glc $ G_1 = (V_1, \Sigma, \mathcal{P}_1, \mathcal{S}_1 ) $, e 
                            $ G_2 = (V_2, \Sigma, \mathcal{P}_2, \mathcal{S}_2 ) $, com $ V_1 \cap V_2 = \Sigma $
                            tq  $ L(G_1) = L_1 $ e $ L(G_2) = L_2 $.
    \begin{itemize}
        \item União: \\ 
                Considere a glc $ G = ( V, \Sigma, \mathcal{P}, \mathcal{S} ) $, onde \\
                $ V = V_1 \cup V_2 \cup \{\mathcal{S}\} $ ( onde $ \mathcal{S} \notin V_1 \cup V_2 ) $ \\
                e $ \mathcal{P} = \{ \mathcal{S} \rightarrow \mathcal{S}_1 | \mathcal{S}_2 \} \cup \mathcal{P}_1 \cup 
                                    \mathcal{P}_2 $.\\
                Prove que $ L(G) = L(G_1) \cup L(G_2) $.
        \item Concatenação: \\
                Considere a glc $ G = ( V, \Sigma, \mathcal{P}, \mathcal{S} ) $, onde \\
                $ V = V_1 \cup V_2 \cup \{\mathcal{S}\} $ ( onde $ \mathcal{S} \notin V_1 \cup V_2 ) $ \\
                e $ \mathcal{P} = \{ \mathcal{S} \rightarrow \mathcal{S}_1\mathcal{S}_2 \} \cup \mathcal{P}_1 \cup 
                                    \mathcal{P}_2 $. \\
                Prove que $ L(G) = L(G_1)L(G_2) $.
        \item Estrela: \\
                Considere a glc $ G = ( V, \Sigma, \mathcal{P}, \mathcal{S} ) $, onde \\
                $ V = V_1 \cup \{\mathcal{S}\} $ ( onde $ \mathcal{S} \notin V_1 ) $ \\
                e $ \mathcal{P} = \{ \mathcal{S} \rightarrow \lambda|\mathcal{S}_1\mathcal{S} \} \cup \mathcal{P}_1 $.\\
                Prove que $ L(G) = (L(G_1))^* $.
    \end{itemize}

\section{Autômatos com pilha}
    \begin{itemize}
        \item Um autômato com pilha não-determinístico é um sistema
                $\mathcal{A} = ( \mathcal{Q}, \Sigma, \Gamma, \delta, s, F ) $:
                \begin{itemize}
                    \item $\mathcal{Q}$ é o conjunto finito de \underline{estados};
                    \item $\Sigma$ é o alfabeto dos \underline{símbolos de entrada};
                    \item $\Gamma$ é o alfabeto dos \underline{símbolos da pilha};
                    \item $ s \in \mathcal{Q} $ é o \underline{estado inicial};
                    \item $ F \subseteq \mathcal{Q} $ é o conjunto de \underline{estados finais};
                    \item $ \delta : \mathcal{Q} \times ( \Sigma \cup \{\lambda\} ) \times ( \Gamma \cup \{\lambda\} ) 
                            \longrightarrow 2^{\mathcal{Q} \times ( \Gamma \cup \{\lambda\} ) } $ é a 
                            \underline{função de transição}.
                \end{itemize}
    \item Uma computação num a.p.n-det. começa no estado inicial, com a entrada na fita e a pilha vazia.
    \item Seja $ \mathcal{A} = ( \mathcal{Q}, \Sigma, \Gamma, \delta, s, F ) $ um a.p.n-det.\\
            Se $(q, \mathcal{B} ) \in \delta(p,a,\mathcal{A}) $ 
            ( $q =$ próximo estado, $\mathcal{B} =$ símbolo a ser empilhado ou $\lambda$,
              $p =$ estado atual, $a =$ simbolo da entrada ou $\lambda$, $A =$ simbolo no topo da pilha ou $\lambda$ ).
    \item Uma configuração de $\mathcal{A}$ é um elemento de $ \mathcal{Q} \times \Sigma^* \times \Gamma^* $ descrevendo:
        \begin{itemize}
            \item O estado inicial;
            \item A parte não lida da entrada;
            \item O conteúdo atual da pilha ( topo..base).
        \end{itemize}
    \item A configuração inicial para uma entrada $x$ deve ser $(s,x,\lambda)$.
    \end{itemize}
    
    $\vdash_\mathcal{A}$ relaciona duas configurações consecutivas de $\mathcal{A}$.\\
    Ex: se $ (q,B) \in \delta(p,a,A)$, com $ a \in (\Sigma \cup \{\lambda\} ) $ e $ A, B \in (\Gamma \cup \{\lambda\} ) $,
        então para qualquer $y \in \Sigma^* $ e $ \gamma \in \Gamma^* $, $ ( p, ay, A\gamma ) \vdash_\mathcal{A} (q, y, B
        \gamma )$.
    
    \begin{itemize}
        \item $\vdash_\mathcal{A}^*$ denota o fecho reflexivo e transitivo de $\vdash_\mathcal{A}$.
        \item Uma palavra $x \in \Sigma^*$ é aceita por $\mathcal{A}$ se existe uma computação tq\\
            $ ( s, x, \lambda ) \vdash_\mathcal{A}^* ( q, \lambda, \lambda ) $ para algum $ q \in F$
        \item A linguagem reconhecida por $\mathcal{A}$ por estado final e pilha vazia é:\\
            $ L(\mathcal{A}) = \{ x \in \Sigma^* : x $ é aceita por $ \mathcal{A} \} $.
    \end{itemize}
    
    \begin{itemize}
        \item [\underline{Exemplo 1}:] $ L = \{ a^nb^n: n \geq 0 \} $
    \end{itemize}
\end{document}
